$-- You can use as many custom partials as you need. Convention is to prefix name with '_'
$-- It can be useful to use such template to split some template parts in smaller pieces, which is easier to reuse.
$-- This '_custom.tex' is used on 'title.tex' as example.
$-- See other existing format in quarto-journals/ organisation.
$-- %%%% TODO %%%%%
$-- Use it if you need to insert content at this specific place of the main Pandoc's template. Otherwise, remove it.
$-- Here we are using it to format the authors part of the template.
$-- %%%%%%%%%%%%%%%


$for(by-author)$
  \author{$by-author.name.literal$}
  $if(by-author.email)$
    $if(by-author.attributes.corresponding)$
    \email{$by-author.email$}
    $endif$
  $endif$
  $if(by-author.affiliations)$
    $for(by-author.affiliations)$
      \affiliation{%
        $if(it.name)$
          $it.name$
        $endif$
        $if(it.address)$
          , $it.address$
        $endif$
        $if(it.city)$
          , $it.city$
        $endif$
        $if(it.state)$
          , $it.state$
        $endif$
        $if(it.country)$
          , $it.country$
        $endif$
        $if(it.postal-code)$
          , $it.postal-code$
        $endif$
      }
    $endfor$
    $for(by-author.affiliations/rest)$
      \alsoaffiliation{%
        $if(it.name)$
          $it.name$
        $endif$
        $if(it.address)$
          , $it.address$
        $endif$
        $if(it.city)$
          , $it.city$
        $endif$
        $if(it.state)$
          , $it.state$
        $endif$
        $if(it.country)$
          , $it.country$
        $endif$
        $if(it.postal-code)$
          , $it.postal-code$
        $endif$
      }
    $endfor$
  $endif$
$endfor$


$--$it.name.literal$$if(it.orcid)$~\orcidlink{$it.orcid$}$endif$$for(it.affiliations/first)$\\$it.name$$endfor$\\$if(it.email)$\href{mailto:$it.email$}{$it.email$}$endif$
